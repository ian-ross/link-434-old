\documentclass[a4paper,11pt]{article}

\usepackage[utf8]{inputenc}
\usepackage{fourier}
\usepackage{amsmath,amssymb}
\usepackage{xstring,ifthen,xcolor}
\usepackage{xspace}
\usepackage{url}

\usepackage{color}
\definecolor{orange}{rgb}{0.75,0.5,0}
\definecolor{magenta}{rgb}{1,0,1}
\definecolor{cyan}{rgb}{0,1,1}
\definecolor{grey}{rgb}{0.25,0.25,0.25}
\newcommand{\outline}[1]{{\color{grey}{\scriptsize #1}}}
\newcommand{\todo}[1]{{\color{red}\textit{\textbf{#1}}}}
\newcommand{\note}[1]{{\color{blue}\textit{\textbf{#1}}}}
\newcommand{\citenote}[1]{{\color{orange}{[\textit{\textbf{#1}}]}}}

\usepackage{tikz}
\usepackage{pgfmath}
\usetikzlibrary{calc,shapes,positioning}

\title{Link-434: EFM8UB3 assembly programming notes}
\author{Ian~Ross}

\graphicspath{{figs/}}

\begin{document}

\maketitle

These notes record things I've been picking up while experimenting
with assembly programming on the Silicon Labs EFM8UB3.

\section{Clock and I/O setup}

The EFM8UB3 has two high-speed oscillators, \texttt{HFOSC0}
(24.5\,MHz) and \texttt{HFOSC1} (48\,MHz). I've had trouble getting
the SEGGER J-Link software to work with the EFM8 development board
when using \texttt{HFOSC1} so for the moment I'm using the 24.5\,MHz
oscillator. \emph{I'll need to get the 48\,MHz oscillator working for
USB applications later on!}

It's actually a little bit strange, because an application built using
Simplicity Studio works fine with \texttt{HFOSC1}, which indicates
that I'm doing something wrong somehow.

\textbf{Get this working!}

\section{Basic code patterns and macros}

\begin{itemize}
  \item{The \texttt{sdas8051} assembler that's part of the SDCC
    distribution has some macro capabilities that make writing
    assembly code a bit easier, but it doesn't support all of the
    features documented in the SDAS manual. In particular, there's no
    \texttt{ifnb} or similar, which makes writing variadic macros
    difficult (impossible?).}
\end{itemize}

\section{Timers}

\emph{Have successfully got something working using interrupts and
timer 0 running off the 48 MHz HFOSC1 oscillator. Next step is to tidy
that up a little and think about how to structure the whole ``Simon
Says'' code.}

\end{document}
